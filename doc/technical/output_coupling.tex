% ICON
%
% ------------------------------------------
% Copyright (C) 2004-2024, DWD, MPI-M, DKRZ, KIT, ETH, MeteoSwiss
% Contact information: icon-model.org
% See AUTHORS.TXT for a list of authors
% See LICENSES/ for license information
% SPDX-License-Identifier: CC-BY-4.0
% ------------------------------------------

\documentclass{article}

\usepackage{hyperref}
\usepackage[top=1in, bottom=1.25in, left=1.25in, right=1.25in]{geometry}


\begin{document}

\section*{Concept description of the output coupling}
\textit{Author 01/2024: N.\ Dreier, DKRZ}

The ``output coupling'' is a generic interface for output components
using the coupler YAC.
It allows users to develop their own taylored output components
independent of ICON.
If the output coupling is enabled in the namelist
\mbox{(\texttt{\&coupling\_mode\_nml\%coupled\_to\_output = .TRUE.})}
ICON calls all \mbox{\texttt{yac\_fdef\_field}} for all proper (see below)
variables in the ICON varlists and stores the \texttt{field\_id} in a
\texttt{field\_list}.
Furthermore some metadata (currently only the CF metadata) is attached
to the field.
3D variables are also exposed with a collection size corresponding to
the vertical axis.
After the definition phase of YAC (after \mbox{\texttt{yac\_fenddef}})
all fields are removed from the list that are not coupled.
In the timeloop then for all remaining fields \texttt{yac\_fput} with
the corresponding data memory pointer is called.
The output coupling is available in the atmosphere and ocean components.

Variables are filtered based on the following criteria:
\begin{itemize}
\item The flag \texttt{info\%loutput} must be set.
\item Variables that are infact a container of tracers are not exposed.
\item The horizontal grid must either be \mbox{\texttt{GRID\_UNSTRUCTURED\_CELL}} or \mbox{\texttt{GRID\_UNSTRUCTURED\_VERT}}
\item The first dimension of the variable must be \texttt{nproma}.
\item For 2d variables: The 2\textsuperscript{nd} dimension must be \texttt{nblks}
\item For 3d variables: The 3\textsuperscript{rd} dimension must be \texttt{nblks}
\end{itemize}

\textbf{Caveat}: By this approach it is currently not ensured that the
variables contain proper values. Some diagnstics are only computed for
the namelist output and hence are not updated if no namelist output is
attached to the variable. Hence this variable would deliver invalid
values over this interface. A workaround is to configure a dummy
namelist output that does output in an timespan outside of the runtime
of the simulation.

\subsection*{References}
\begin{itemize}
\item The YAC documentation: \url{https://dkrz-sw.gitlab-pages.dkrz.de/yac/}
\end{itemize}


\end{document}
