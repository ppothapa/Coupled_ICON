% ICON
%
% ------------------------------------------
% Copyright (C) 2004-2024, DWD, MPI-M, DKRZ, KIT, ETH, MeteoSwiss
% Contact information: icon-model.org
% See AUTHORS.TXT for a list of authors
% See LICENSES/ for license information
% SPDX-License-Identifier: CC-BY-4.0
% ------------------------------------------

\chapter{Special treatment for the triangular grid}

We encountered widespread difficulties with the triangular C-grid. They can be cast under
different statements
\begin{itemize}
 \item The horizontal divergence operator turns out to be only first order accurate and exhibits a
checkerboard error pattern. The error is especially pronounced for deformational flow fields.
 \item The simple kinetic energy has to be composed in a way to avoid the Hollingsworth instability.
\end{itemize}
Trying to bypass those problems, obviously the first attempt is to create a second order horizontal
divergence operator, which is achieved by averaging the divergences. This would lead to
the renouncement of the nice C-grid properties, namely the non-zero frequency for the
shortest resolvable traveling gravity waves. Grid scale noise can easily amplify, because
no operator 'sees' it. Unfortunately, we can not see this, because the noise is in the grid scale 
wind field, which is not a direct output variable. RBF-interpolation of the wind vector for output 
masks the problem. Thus, we are in an obvious dilemma, as the triangular grid was our strategy during 
the whole development time of the ICON model, because it also delivered some advantages
\begin{itemize}
 \item Before Thuburn (2008) solved the problem of the dispersion relation for the hexagonal grid, 
no way was seen to use an alternative grid structure, although heavily investigated in the thesis of
William Swayer (2006) under the supervision of Luca Bonaventura.
 \item The triangular grid seems to be best suited for grid refinement.
 \item There were already other small scale ocean models (but not a global one) around that 
successfully used the triangular C-grid .
\end{itemize}
Although the main problems were finally solved for the alternative hexagonal grid, there was not yet
enough time to investigate it's capabilities in the vicinity of boundaries, which is
essential for the ocean, and the strategy for grid refinement was not yet tackled.
The hexagonal grid, at least, does not show problems with energy conservation, as
straightforward discretisations for divergence and gradient are both of second order
accuracy.

Thus, we must live with the triangular grid and hide its deficiencies as far as possible. 

After a long time of experimentation with the hydrostatic model,
a mix of smoothing methods for the horizontal divergence was eventually found.

We use a bilinear averaging on 4 triangles to remove the checkerboard
in the divergence. Because the bilinear averaging removes mass consistency,
an iterative procedure is taken to restore it again.
The same averaging procedure might serve as to average inner products
that naturally occur on triangles. That occurs for the evaluation
of the kinetic energy and the metric correction terms in the contravariant
vertical velocity and the orthogonal vertical vorticity.

Another measure to remove noise in model fields, is to apply diffusion
to the normal velocity equation as Hui Wan showed in her thesis (2009),
the numerical error in the 4rth order Laplacian might serve as to
remove the checkerboard noise in each time step when it is produced.
Unfortunately, the required diffusion coefficient is such that the
characteristic damping time is of the order of one time step. Even though until now
no degradation of the model results seems to show up, the amount of
smoothing needed to keep the model results stable, stays alarming,
and we do not yet know how the model reacts if full physics comes into play.
