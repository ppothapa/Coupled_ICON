% &&&&&&&&&&&&&&&&&&&&&&&&&&&&&&&&&&&&&&&&&&&&&&&&&&&&&&&&&&&&&&&&&&&&&&&&&&&&&
%
% ICON_WAVES_NML_TABLES.TEX
%
% This LaTeX file defines ocean waves specific namelist parameters.
% The file is included by ICON_NML_TABLES.TEX. For atmosphere-only releases
% the ocean waves namelists can be excluded.
%
% &&&&&&&&&&&&&&&&&&&&&&&&&&&&&&&&&&&&&&&&&&&&&&&&&&&&&&&&&&&&&&&&&&&&&&&&&&&&&

\section{Ocean waves specific namelist parameters}

%------------------------------------------------------------------------------
% energy_propagation_nml:
%------------------------------------------------------------------------------
\subsection{energy\_propagation\_nml (used if run\_nml/ltransport=.TRUE.)}

\begin{longtab}

%\hline
itype\_limit&
I&
0& & Type of limiter for wave energy transport:& \tabularnewline
& & & & 0: no limiter& \tabularnewline
& & & & 3: monotonic flux limiter (FCT)& \tabularnewline
& & & & 4: positive definite flux limiter& \tabularnewline


%\hline
beta\_fct&
R& 1.005& & global boost factor for range of permissible values $\left[q_{max},q_{min}\right]$ in the monotonic flux limiter. A value larger
than 1 allows for (small) over and undershoots, while a value of $1$ gives strict monotonicity (at the price of increased diffusivity). &
itype\_limit = 3
\tabularnewline


%\hline
igrad\_c\_miura&
I& 1& & Method for gradient reconstruction at cell center for 2nd order miura scheme& \tabularnewline
& & & & 1: Least-squares (linear, non-consv)& \tabularnewline
& & & & 2: Green-Gauss& \tabularnewline
& & & & 3: based on shape function derivatives for a three-node triangular element (Fish. J and T. Belytschko, 2007) &
\tabularnewline

%\hline
lgrid\_refr     &
L& .TRUE. &     & .TRUE.: calculate grid refraction & 
\tabularnewline
%\hline

\end{longtab}

Defined and used in: \verb+src/waves/config/mo_energy_propagation_nml.f90+



\subsection{waves\_nml (used if configurated with --enable-waves and model\_type=98 in master\_model\_nml)}

\begin{longtab}
%Parameter &Type& Default   & Unit& Description&                                                      Scope
 ndirs     & I  & 24        &     & number of direction of wave spectrum & \tabularnewline
 nfreqs    & I  & 25        &     & number of frequencies of wave spectrum & \tabularnewline
 fr1       & R  & 0.04177248& Hz  & first frequency of wave spectrum & \tabularnewline
 co        & R  & 1.1       &     & frequency ratio & \tabularnewline
 iref      & I  & 1         &     & frequency bin number of reference frequency & \tabularnewline
 alpha     & R  & 0.018     &     & Phillips parameter & \tabularnewline
 fm        & R  & 0.2       & Hz  & peak frequency and/or maximum frequency & \tabularnewline
 gamma\_wave&R  & 3.0       &     & overshoot factor & \tabularnewline
 sigma\_a   &R  & 0.07      &     & left peak width of wave spectrum & \tabularnewline
 sigma\_b   &R  & 0.09      &     & right peak width of wave spectrum & \tabularnewline
 fetch     & R  & 300000.   &  m  & fetch & \tabularnewline
 roair     & R  & 1.225     &kg/m3& air density & \tabularnewline
 rnuair    & R  & 1.5e-5    &m2/s & kinematic air viscosity & \tabularnewline
 rnuairm    & R  & 0.11*rnuair&m2/s& kinematic air viscosity for momentum transfer & \tabularnewline
 rowater   & R  & 1000.     &kg/m3& water density & \tabularnewline
 xeps      & R  & roair/rowater&  & air water density ratio & \tabularnewline
 xinveps   & R  & 1./xeps   &     & insersed air water density ratio & \tabularnewline
 betamax   & R  & 1.20      &     & parameter for wind input (ECMWF cy45r1) & \tabularnewline
 zalp      & R  & 0.0080    &     & shifts growth curve (ECMWF cy45r1) & \tabularnewline
 jtot\_tauhf& I  & 19       &     & dimension of high freuency wave stress (wtauhf) & must be odd \tabularnewline 
 alpha\_ch  & R  & 0.0075   &     & minimum Charnock constant (ecmwf cy45r1) & \tabularnewline
 depth     &  R & 0.        & m   & ocean depth if not 0, then constant depth & \tabularnewline
 niter\_smooth  &  I & 1    &     & number of smoothing iterations for wave bathymetry & \tabularnewline
 xkappa    &  R & 0.40      &     & von Karman constant & \tabularnewline
 xnlev     &  R & 10.0      & m   & windspeed reference level & \tabularnewline
 linput\_sf1& L  & .TRUE.   &     & .TRUE.: calculate wind input source function term & \tabularnewline
 linput\_sf2& L  & .TRUE.   &     & .TRUE.: update wind input source function term & \tabularnewline
 ldissip\_sf& L  & .TRUE.   &     & .TRUE.: calculate dissipation source function term & \tabularnewline
 lnon\_linear\_sf&L & .TRUE.&     & .TRUE.: calculate non linear source function term & \tabularnewline
 lbottom\_fric\_sf&L& .TRUE.&     & .TRUE.: calculate bottom friction source function term & \tabularnewline
 lwave\_stress1  &L& .TRUE. &     & .TRUE.: calculate wave stress & \tabularnewline
 lwave\_stress2  &L& .TRUE. &     & .TRUE.: update wave stress & \tabularnewline
 forc\_file\_prefix &C&     &     & common prefix of forcing files & \tabularnewline
                    & &     &     & if not empty, the names of forcing files will be consctructed as: & \tabularnewline
                    & &     &     & forc\_file\_prefix+\_wind.nc - for 10m wind & coupled\_mode= .FALSE. in coupling\_mode\_nml \tabularnewline
                    & &     &     & forc\_file\_prefix+\_ice.nc - for sea ice concentration & coupled\_mode= .FALSE. in coupling\_mode\_nml \tabularnewline
                    & &     &     & forc\_file\_prefix+\_slh.nc - for sea level height & \tabularnewline
                    & &     &     & forc\_file\_prefix+\_osc.nc - for ocean surface currents & \tabularnewline
                    & &     &     & Data for all time steps in the current simulation should be prepared in a single file. Variables should be named \texttt{u\_10m}, \texttt{v\_10m}, \texttt{fr\_seaice}, \texttt{uosc}, \texttt{vosc} & \tabularnewline
\end{longtab}

Defined and used in: \verb+src/waves/config/mo_wave_nml.f90+
