\documentclass[11pt]{article}
\usepackage[a4paper,margin=1.5cm]{geometry}
\textwidth=18cm
% \oddsidemargin=0cm
\usepackage{graphicx}
\usepackage{amssymb}
\usepackage[pdftex,colorlinks=true, breaklinks=true, linkcolor=blue]{hyperref} 
\usepackage{natbib}
\usepackage{booktabs}
\usepackage{multirow}
\usepackage[table]{xcolor}
\usepackage{array}

\title{ICON model parameters suitable for model tuning}
\author{Linda Schlemmer, G\"unter Z\"angl, J\"urgen Helmert, Martin K\"ohler, Dmitrii Mironov,\\
   Matthias Raschendorfer, Daniel Reinert, Daniel Rieger, Sophia Sch\"afer, Axel Seifert \\ \\ Deutscher Wetterdienst (DWD), Offenbach}
\date{\today}                      
\begin{document}
\maketitle

The table summarizes the most important tuning variables for the ICON model, and is largely based on \citet{ICON_database}, chapter 12.2. Those parameters that have been identified as sensitive parameters in \citet{cosmo_42} are highlighted in green. Parameters that are mostly relevant for simulations covering the tropics are highlighted in blue.\\

Yet, the document and the list of variables should be handled with care. Purely varying some of the listed parameters blindly will most likely not give satisfactory results. A physical understanding of the identified model shortcomings/biases should be built up first, followed by a choice of the associated model parameters and a systematic variation and evaluation of simulations. The parameters of interest may strongly vary for the region of interest, the model resolution and the specific purpose.\\

Please also keep in mind that the list is neither exhaustive, nor complete. There may well be further model parameters that are more suitable for individual applications.

% SSO and GWD

   \begin{center}
   \begin{tabular}{>{\raggedright}p{0.15\linewidth}p{0.35\linewidth}p{0.18\linewidth}p{0.23\linewidth}} 
     \toprule
      Parameter    & Description & Meaningful Range& Comment\\
     \midrule
      \rowcolor[rgb]{1.,0.8,0.8}\multicolumn{4}{c}{Tuning of the SSO and GWD parameters is dependent on the employed external paramters}\\ \cmidrule{1-4}
      \multicolumn{2}{c}{\bf SSO tuning}\\ \cmidrule{1-2}
      gkwake     & low level wake drag constant $C_d$ for blocking& 1.5 $\pm$ 0.5 & Very strong dependency on raw data resolution: for ICON-D2 with ASTER data, we use 0.25\\
      gkdrag      & gravity wave drag constant $G$, a function of
                          mountain sharpness & 0.075 $\pm$ 0.04 & Should be zero (turned off) at convection-permitting resolutions\\
      gfrcrit        & critical Froude number determining depth of blocked layer $H_{n_{crit}}$  & 0.4 $\pm$ 0.1&\\
      grcrit         & critical Richardson number & 0.25 & \\
      tune\_minsso & minimal value of SSO-STDH (m) where SSO-effects are being considered & default 10 & must also be adapted in extpar!\\
      tune\_blockred & multiples of the SSO-STDH, above which the SSO-blocking tendency is being reduced proportionally to STDH/z\_AGL & 1.5 & default 100 = deactivated \\
      \addlinespace[10pt]
      \multicolumn{2}{c}{\bf GWD tuning}\\ \cmidrule{1-2}
      gfluxlaun&variability range for non-orographic gravity wave launch momentum flux &2.50$\cdot 10^{-3} \newline  \pm  0.75 \cdot 10^{-3}$ [Pa] &relevant for global applications only\\
     \bottomrule
   \end{tabular}
   \end{center}

% microphysics and cloud cover

   \begin{center}
   \begin{tabular}{p{0.15\linewidth}p{0.35\linewidth}p{0.18\linewidth}p{0.23\linewidth}}  
     \toprule
      Parameter    & Description & Meaningful Range& Comment\\
     \midrule
     \multicolumn{2}{c}{\bf grid scale microphysics}\\ \cmidrule{1-2}
      \rowcolor[rgb]{0.8,0.8,1.}zvz0i   & terminal fall velocity of ice & 0.85 $\pm$ 0.25 [m/s] 
         &allows temperature bias tuning in the upper tropical troposphere as well as TOA long-wave fluxes\\
      \rowcolor[rgb]{0.8,0.8,1.}zceff\_min  & minimum value for sticking efficiency& 0.01 - 0.075 &tropics\\
      \rowcolor[rgb]{0.8,1,0.8}v0snow    & factor in the terminal velocity for snow & 10.0 - 30.0 & recommended value 25.0\\
      icesedi\_exp & exponent for density correction of cloud ice sedimentation & 0.3 -  0.33 &no perturbation recommended \\
      \rowcolor[rgb]{0.8,1,0.8}rain\_n0fac & multiplicative change of intercept parameter of raindrop size distribution & 0.25 - 4. & multiplicative perturbation \\
     \addlinespace[10pt] 
     \multicolumn{2}{c}{\bf cloud cover}\\ \cmidrule{1-2}
      box\_liq   & Box width for liquid cloud diagnostic in cloud cover scheme & 0.05 $\pm$ 0.02 &\\
      box\_liq\_asy & Asymmetry factor for liquid cloud cover diagnostic & 2.0 - 4.0 (def.\ 3.25)& sensitive to TOA solar fluxes and to a lesser degree long-wave fluxes\\
      thicklayfac & factor for increasing the box width for layer thicknesses exceeding 150\,m & 0.005 $\pm$ 0.005 [1/m] & accounting for vertical sub-grid overlap\\
      sgsclifac & Scaling factor for turbulence-induced subgrid-scale contribution to diagnosed cloud ice & 0.0 - 1.0 & 0.0 turns this contribution off\\
      allow\_overcast & Tuning factor for steeper dependence CLC (RH) & $\leq$ 1.0 & setting allow\_overcast$<$1 together with reduction of tune\_box\_liq\_asy causes steeper CLC(RH) dependence. {\color{red}{recommendation: allow\_overcast$<$1 should not be used in combination with lsgs\_cond=.TRUE.}}\\
     \bottomrule
   \end{tabular}
   \end{center}

% turbulence

   \begin{center}
   \begin{tabular}{p{0.15\linewidth}p{0.35\linewidth}p{0.18\linewidth}p{0.23\linewidth}}  
   \toprule
      Parameter    & Description & Meaningful Range& Comment\\
      \midrule
      \multicolumn{2}{c}{\bf turbulence}\\ \cmidrule{1-2}
      \rowcolor[rgb]{0.8,1,0.8}q\_crit & critical value for normalised super-saturation & 1.6-4.0& \\
      \rowcolor[rgb]{0.8,1,0.8}rlam\_heat & scaling factor of the laminar boundary layer for heat (scalars), the change in rlam\_heat is
         accompanied by an inverse change of rat\_sea in order to keep the evaporation over water (controlled by rlam\_heat $\cdot$ rat\_sea) the same. {\color{red}{recommendation: the product of rlam\_heat and rat\_sea should not be significantly larger than 10. Otherwise, there will be too little evaporation over the oceans.}}
         & 10.0$\pm$8.0 &additive perturbation\\
      \rowcolor[rgb]{0.8,1,0.8}rat\_sea & controls latent and sensible heat fluxes over water
         & 0.8 - 10.0 & lower values increase latent and sensible fluxes over water; different values in data\_turbulence.f90 and turb\_data.f90 ? \\
      a\_hshr & Length scale factor for the separated horizontal shear mode & 1.0 $\pm$ 1.0 & \\
      \rowcolor[rgb]{0.8,1,0.8}a\_stab & factor for stability correction of turbulent length scale & 0.0 & turned off by default because it degrades global skill scores\\
      c\_diff & length scale factor for vertical diffusion of TKE & 0.1-0.4 &   \\
      alpha0 & lower bound of velocity-dependent Charnock parameter & 0.0123-0.0335 & additive ensemble perturbation of Charnock-parameter \\
      alpha1 & parameter scaling the molecular roughness of water waves & 0.1-1.0 & lower values increase latent and sensible fluxes over water, particularly at low wind speeds. alpha1=1.0 in data\_turbulence.f90 and alpha1=0.75 in turb\_data.f90, recommended value of 0.125\\
      \rowcolor[rgb]{0.8,1,0.8}tur\_len & asymptotic maximal turbulent distance &500. alpha$\pm$ 150. [m] & \\
      \rowcolor[rgb]{0.8,1,0.8}tkhmin & scaling factor for minimum vertical diffusion coefficient for heat and moisture & 0.6 $\pm$ 0.2 & 0.75 default in code\\
      \rowcolor[rgb]{0.8,1,0.8}tkmmin & scaling factor for minimum vertical diffusion coefficient for momentum & 0.75 $\pm$ 0.2 & \\
      tkred\_sfc & multiplicative change of reduction of minimum diffusion coefficients near the surface & 0.25 - 4.0 & multiplicative perturbation \\
   \bottomrule
   \end{tabular}
   \end{center}
   
% TERRA, snow and radiation

   \begin{center}
   \begin{tabular}{p{0.15\linewidth}p{0.35\linewidth}p{0.18\linewidth}p{0.23\linewidth}}  
   \toprule
      Parameter    & Description & Meaningful Range& Comment\\
      \midrule
      \multicolumn{2}{c}{\bf TERRA}\\ \cmidrule{1-2}
      \rowcolor[rgb]{0.8,1,0.8}c\_soil &  evaporating fraction of soil & 1.0 $\pm$ 0.25 & \\
      cwimax\_ml & scaling parameter for maximum interception storage & 5.$\cdot 10^{-7} -  5.\cdot 10^{-4}$ & low values ($<10^{-6}$) turn off interception layer\\
      \addlinespace[10pt]
      \multicolumn{2}{c}{\bf snow cover diagnosis}\\ \cmidrule{1-2}
      minsnowfrac & Lower limit of snow cover fraction to which melting snow is artificially reduced in the context of the snow-tile approach & 0.2 $\pm$ 0.1 & \\
      \addlinespace[10pt]
      \multicolumn{2}{c}{\bf radiation}\\ \cmidrule{1-2}
      dust\_abs & Tuning factor for enhanced LW absorption of mineral dust in the Saharan region & 0.0 & Reduces bias over Sahara for the RRTM scheme but not necessary and implemented with ecRad and itype\_lwemiss=2\\
   \bottomrule
   \end{tabular}
   \end{center}

% convection

   \begin{center}
   \begin{tabular}{p{0.15\linewidth}p{0.35\linewidth}p{0.18\linewidth}p{0.23\linewidth}}  
   \toprule
      Parameter    & Description & Meaningful Range& Comment\\
      \midrule
      \multicolumn{2}{c}{\bf convection}\\ \cmidrule{1-2}
      entrorg & Entrainment parameter in convection scheme valid for dx=20km & 1.95$\cdot 10^{-3}\pm 0.2\cdot 10^{-3}$ 
        & corresponds to entr\_sc in the shallow convection part of COSMO Tiedtke scheme\\
      rdepths & maximum allowed shallow convection depth & 2.0 $\cdot 10^{4}\pm 5.0\cdot 10^{3}$ Pa& \\
      rprcon & coefficient for conversion of cloud water into precipitation & 1.4$\cdot 10^{-3}\pm 0.25\cdot 10^{-3}$ & \\
      capdcfac\_et & fraction of CAPE diurnal cycle correction applied in the extratropics & 0.5 $\pm$ 0.75 & \\
      capdcfac\_tr & fraction of CAPE diurnal cycle correction applied in the tropics & 0.5 $\pm$ 0.75 & \\
      lowcapefac & Tuning parameter for diurnal-cycle correction in convection scheme: reduction factor for low-cape situations & 1.0 $\pm$ 0.5 & \\
      negpblcape & Tuning parameter for diurnal-cycle correction in convection scheme: maximum negative PBL CAPE allowed in the modified CAPE closure & -500.- 0.\\
      rhebc\_land & RH threshold for onset of evaporation below cloud base over land & 0.825 $\pm 0.05$ & 0.75 as default in code\\
      rhebc\_ocean & RH threshold for onset of evaporation below cloud base over sea & 0.85 $\pm 0.05$ & \\
      \rowcolor[rgb]{0.8,0.8,1.}rhebc\_land\_trop & RH threshold \dots over tropical land & 0.70 $\pm 0.05$ & \color[rgb]{0,0,1}tropics\\
      \rowcolor[rgb]{0.8,0.8,1.}rhebc\_ocean\_trop & RH threshold \dots over tropical sea & 0.76 $\pm 0.05$ & \color[rgb]{0,0,1}tropics\\
      rcucov & Convective area fraction used for computing evaporation below cloud base & 0.075 & 0.05 coded as default\\
      \rowcolor[rgb]{0.8,0.8,1.}rcucov\_trop & Convective area fraction used for computing evaporation below cloud base, tropics & 0.03 &\color[rgb]{0,0,1}tropics\\
      texc & Excess value for temperature used in test parcel ascent & 0.125 $\pm$ 0.05 [K] & \\
      qexc & Excess fraction of grid-scale QV used in test parcel ascent &  0.0125 \newline $\pm$ 0.005 [kg/kg]& \\
   \bottomrule
   \end{tabular}
   \end{center}


\bibliographystyle{ametsoc2014}
\bibliography{icon_tuning_vars}

\end{document}  
